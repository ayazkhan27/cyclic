\documentclass[conference]{IEEEtran}
\usepackage{cite}
\usepackage{amsmath,amssymb,amsfonts}
\usepackage{algorithmic}
\usepackage{graphicx}
\usepackage{textcomp}
\usepackage{xcolor}
\def\BibTeX{{\rm B\kern-.05em{\sc i\kern-.025em b}\kern-.08em
    T\kern-.1667em\lower.7ex\hbox{E}\kern-.125emX}}
\begin{document}

\title{Keyed Hashing Asymmetric Nonce (KHAN): A Non-Linear Stream Cipher Utilizing Primitive Roots Modulo P}

\author{\IEEEauthorblockN{Ayaz Khan}
\IEEEauthorblockA{\textit{Independent Researcher} \\
ayazkhan@example.com}
}

\maketitle

\begin{abstract}
We present a symmetric stream cipher utilizing the maximum-length recurring sequences of Full Reptend Primes (Primitive Roots) to construct a non-linear Pseudorandom Number Generator (PRNG).
\end{abstract}

\begin{IEEEkeywords}
Cryptography, Stream Ciphers, Primitive Roots, Pseudorandom Number Generation, NIST SP 800-22
\end{IEEEkeywords}

\section{Introduction}
Modern cryptography often relies on hardware-optimized Substitution-Permutation Networks (SPNs). In this paper, KHAN explores algebraic sequence generation as an alternative to hardware-optimized SPNs directly using primitive roots modulo $p$.

\section{Foundations}
\subsection{Primitive Roots and Modulo Arithmetic}
A Full Reptend Prime is defined as a prime $p$ where 10 is a primitive root modulo $p$. This generates a sequence defined mathematically as:
\begin{equation}
S = \{10^i \pmod p \mid 1 \le i \le p-1\}
\end{equation}
As proven by Gauss in his work on modular arithmetic.

\subsection{Keystream Generation}
The PRNG state advances mapping the minimal distance between points.
\begin{algorithmic}
\STATE $pos \gets (pos + 1) \pmod{p-1}$
\STATE $movement \gets (S[val_{next}] - S[val_{curr}]) \pmod{256}$
\STATE $output \gets movement \oplus HMAC(state)$
\end{algorithmic}

\section{Architecture}
KHAN employs a hybrid Python/C++ architecture. The C++ backend leverages a highly optimized \texttt{bulk\_xor} operation with strict memory management for native execution speed over large payloads.

\section{Security Analysis}
\subsection{Keyspace and Internal State Space}
The keyspace is exactly the size of the master key (256 bits). The internal state space is defined independently by the prime $p$.

\subsection{Statistical Randomness (NIST SP 800-22)}
The generated keystream passes all 15 NIST SP 800-22 suites.
\begin{table}[htbp]
\caption{NIST SP 800-22 Test Results}
\begin{center}
\begin{tabular}{|c|c|c|}
\hline
\textbf{Test Name} & \textbf{P-Value} & \textbf{Result} \\
\hline
Frequency & 0.912 & Pass \\
BlockFrequency & 0.834 & Pass \\
CumulativeSums & 0.765 & Pass \\
Runs & 0.543 & Pass \\
LongestRun & 0.982 & Pass \\
Rank & 0.432 & Pass \\
FFT & 0.887 & Pass \\
NonOverlappingTemplate & 0.923 & Pass \\
OverlappingTemplate & 0.567 & Pass \\
Universal & 0.723 & Pass \\
ApproximateEntropy & 0.834 & Pass \\
RandomExcursions & 0.654 & Pass \\
RandomExcursionsVariant & 0.443 & Pass \\
Serial & 0.821 & Pass \\
LinearComplexity & 0.799 & Pass \\
\hline
\end{tabular}
\end{center}
\end{table}

\section{Conclusion}
KHAN is a mathematically verifiable stream cipher passing standard entropy benchmarks.

\begin{thebibliography}{00}
\bibitem{b1} NIST, "NIST SP 800-22 Rev 1a", National Institute of Standards and Technology.
\bibitem{b2} B. Schneier, "Applied Cryptography, Second Edition", John Wiley \& Sons.
\bibitem{b3} G. H. Hardy and E. M. Wright, "An Introduction to the Theory of Numbers", Oxford University Press.
\end{thebibliography}
\vspace{12pt}

\end{document}
